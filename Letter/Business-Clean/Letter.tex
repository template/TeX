\documentclass[fontsize=10pt, version=last, foldmarks=true,foldmarks=bTP, 
version=last, BCOR=-18.9mm,refline=narrow, enlargefirstpage=true
]{scrlttr2} 

\usepackage[utf8]{inputenc}
\usepackage[ngerman]{babel}
\usepackage{graphicx}
\usepackage[ngerman, num]{isodate}
\usepackage{blindtext}
%Schriftfamilie Helvetica
\usepackage{helvet}
\renewcommand{\familydefault}{\sfdefault}

% Diese Datei enthält die Informationen fürs Layout!
\daymonthsepgerman{} 
\monthyearsepgerman{}{}

\renewcaptionname{ngerman}{\enclname}{Anlage(n)}

\setlength{\textwidth}{160.9mm}
 
\def\briefkopf{\fontsize{7pt}{9pt}\selectfont}
\def\briefkopfname{\vskip 12pt \hskip -10.4pt \bf \fontsize{10pt}{13pt}\selectfont}
\def\briefkopfspace{\vskip 6pt \hskip -10.3pt}
\def\zweib{\fontsize{10pt}{13pt}\selectfont}

\newcommand {\sendername} [1] {\setkomavar{sendername}{#1}}
\newcommand {\senderzusatz} [1] {\setkomavar{senderzusatz}{#1}}
\newcommand {\senderstrasse} [1] {\setkomavar{senderstrasse}{#1}}
\newcommand {\senderplz} [1] {\setkomavar{senderplz}{#1}}
\newcommand {\senderort} [1] {\setkomavar{senderort}{#1}}
\newcommand {\senderland} [1] {\setkomavar{senderland}{#1}}
\newcommand {\senderphone} [1] {\setkomavar{senderphone}{#1}}
\newcommand {\senderfax} [1] {\setkomavar{senderfax}{#1}}
\newcommand {\senderemail} [1] {\setkomavar{senderemail}{#1}}

\newcommand {\ihrezeichen} [1] {\setkomavar{ihrezeichen}{#1}}
\newcommand {\unserezeichen} [1] {\setkomavar{unserezeichen}{#1}}
\newcommand {\durchwahl} [1] {\setkomavar{durchwahl}{#1}}
\newcommand {\datum} [1] {\setkomavar{date}{#1}}
\newcommand {\anlagen} [1] { \vskip 13pt \hskip -10.3pt {\bf Anlage(n)}\\#1}
\newcommand {\absender} [6] {\setkomavar{location}{\briefkopf
#1
% Name hervorgehoben
{\briefkopfname #2}
\briefkopfspace
#3
\briefkopfspace
#4
\briefkopfspace
#5
\briefkopfspace
#6} {}}
\newcommand {\betreff} [1] { \setkomavar{subject}{#1}}
\newcommand {\signatur} [1] { \setkomavar{signature}{#1}}

\renewcommand*{\raggedsignature}{\raggedright}

\KOMAoptions{backaddress=plain}

\newkomavar*[Ihre Zeichen, Ihre Nachricht vom:]{ihrezeichen} 
\newkomavar*[Unsere Zeichen]{unserezeichen} 
\newkomavar*[Durchwahl	]{durchwahl}
\setkomavar*{enclseparator}{\bf Anlage} 

\begin{document} 

\makeatletter
\@setplength{tfoldmarkvpos}{105mm}

\@setplength{firstheadvpos}{11mm}
\@setplength{firstheadwidth}{69.3mm}

\@setplength{lochpos}{18mm}
\@setplength{locwidth}{49mm}
\@setplength{locvpos}{42.4mm}
\@setplength{locheight}{137pt}

\@setplength{toaddrvpos}{45mm}
\@setplength{toaddrhpos}{24.1mm}
\@setplength{toaddrheight}{26mm}

\@setplength{refvpos}{99mm}
\@setplength{refwidth}{136.4mm}
\@setplength{refhpos}{24.1mm}
\@setplength{refaftervskip}{10mm}

\@setplength[2]{subjectaftervskip}{\baselineskip}

\@setplength[3]{sigbeforevskip}{\baselineskip}

\@addtoplength[-1]{locvpos}{7pt}
\makeatother

\setkomavar{backaddress}{\fontsize{5.5pt}{10pt}\selectfont Frank Zisko, Waldweg 1, 01234 Stadt} {}


% Gewünschtes Signet (Farbdruck oder SW) einfach ein-, bzw. auskommentieren.
%\firsthead{\includegraphics[height=16mm]{HEAD.pdf}}
% Vollständige Absenderadresse
% Absender hat 6 Parameter, je einen pro Block
% Erzwungene Zeilenumbrüche mit \\ berücksichtigen!!!

% Auch in style.sty ändern!!!
\absender{Herr}
{Frank Zisko}
{ }
{ \\Waldweg 1\\01234 Stadt}
{Telefon: +49 123 / 123 123-4\\ Mobil: +49 156 / 10 20 30-4}
{frank.zisko@emailadresse.net\\}
 
% Inhalte der Geschäftszeile 
% Nur die Werte in hinteren geschweiften Klammern verändern!
\ihrezeichen{012345.001.01}
\unserezeichen{98089}
\durchwahl{5}
\datum{\today}
 
% Inhalt der Betreffszeile
\betreff{Ich möchte mich ausdrücken}

% Signatur des Briefes (Unterschrift
\signatur{Frank Zisko}

% Empängeradresse 
% ZeilemumbrÃŒche mit \\ erzwingen, sonst steht gesamte Adresse in einer Zeile!
\begin{letter}{
Empfängerfirma\\
Große Straße 5\\ 
% \bf nicht löschen -> PLZ und Stadt sind fett hervorgehoben!
\bf 98765 Dorf}

% Eigentlicher Briefinhalt mit Anrede und Grußformel 
% Anrede
\opening{Sehr geehrte Damen und Herren,}
%Briefinhalt
\blindtext

\vspace*{5ex}

\noindent
Podeste\\
Platz 1: Frank Zisko\\
\noindent
Konto: 123 123 12\\
BLZ: 120 130 140\\


% Grußformel
\closing{Mit freundlichen Grüßen}
% Signatur wird automatich hier eingefÃŒgt!

% Anlagen
% Anlagen durch Zeilenumbruch (\\) separieren! Alternativ kann der auskommentierte KOMA-Script-Befehl fÃŒr den Anhang verwendet werden.
% Jedoch stimmt die Ausgabe nicht mit den Anforderungen des CD Ìberein. Eventuell unterstÌtzen spÀtere KOMA-Versionen dieses Problem.
%\anlagen{Anlage1\\Anlage2...}
%%\encl{Anlage1\\Anlage2...}

\end{letter} 

%
% Hier könnten weitere Inhalte eingefügt werden, z.B. die Anlagen!
% 

\end{document} 
